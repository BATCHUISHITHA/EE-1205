\documentclass{article}
\usepackage{float}
\usepackage{circuitikz}
\usepackage{cite}
\usepackage{amsmath,amssymb,amsfonts,amsthm}
\usepackage{algorithmic}
\usepackage{graphicx}
\usepackage{textcomp}
\usepackage{xcolor}
\usepackage{txfonts}
\usepackage{listings}
\usepackage{amsmath}
\usepackage{enumitem}
\usepackage{mathtools}
\usepackage{gensymb}
\usepackage{comment}
\usepackage[breaklinks=true]{hyperref}
\usepackage{tkz-euclide} 
\usepackage{listings}
\usepackage{gvv}        

%Enumering lower case roman numerals
%\renewcommand{\theenumi}{\roman{enumi}}   
%\renewcommand{\labelenumi}{\theenumi)}


\begin{document}

\title{ FILTER DESIGN ASSIGNMENT}
\author{EE23BTECH11011- Batchu Ishitha$^{*}$% <-this % stops a space
}
\maketitle

\bigskip

\renewcommand{\thefigure}{\theenumi}
\renewcommand{\thetable}{\theenumi}
%\renewcommand{\theequation}{\theenumi}

\section{\textbf{INTRODUCTION}}
We are supposed to design the equivalent FIR and IIR filter realizations for  given filter number.  
This is a bandpass filter whose specifications are available below.

\section{\textbf{Filter Specifications}}
The sampling rate for the filter has been specified as $F_s =  48$ kHz.	If the un-normalized  discrete-time (natural) frequency is F, the corresponding normalized digital filter (angular) frequency is given by $\omega = 2\pi\left(\frac{F}{F_s}\right)$.

\subsection{\textbf{The Digital Filter}}

\begin{enumerate}

\item {\em Passband:}  The passband of filter number $j$, is from \{4 + 0.6(j)\}kHz
to \{4 + 0.6(j+2)\}kHz where 
\begin{align}
    j=\brak{r-11000} \mod \sigma
\end{align}
where $\sigma$ is sum of digits of roll number and $r$ is roll number.\\
\begin{align}
    r&=11011\\
    \sigma  &= 4\\
    j&=3
\end{align}  
Substituting $j = 3$ gives the passband range for our bandpass filter as $5.8$ kHz - $7$ kHz. 
 Hence, the un-normalized discrete time filter passband frequencies are $F_{p1} = 5.8$ kHz
and $F_{p2} = 7$ kHz.  The corresponding normalized digital filter passband frequencies are

\begin{align}
\omega_{p1} = 2\pi\frac{F_{p1}}{F_s}  = 0.24\pi  \\
 \omega_{p2} = 2\pi\frac{F_{p2}}{F_s}  = 0.29\pi
\end{align}  
 The centre frequency is then given by  
 \begin{align}
 \omega_c = \frac{\omega_{p1} + \omega_{p2}}{2} = 0.265\pi.  
 \end{align}

\item {\em Tolerances:}  The passband ($\delta_1$) and stopband ($\delta_2$) tolerances are given to
be equal, so we let $\delta_1 = \delta_2 = \delta = 0.15$.

\item {\em Stopband:}  The {\em transition band} for bandpass filters is $\Delta F = 0.3$ kHz on either side of the passband.
Hence, the un-normalized {\em stopband} frequencies are  
\begin{align}
F_{s1} = 5.8- 0.3 = 5.5 kHz \\
F_{s2} = 7 + 0.3 = 7.3 kHz .
 \end{align} 
The corresponding normalized frequencies are
\begin{align}
\omega_{s1} =  0.2292\pi \\
\omega_{s2} = 0.3041 \pi.
\end{align}
\end{enumerate}

\subsection{\textbf{The Analog filter}} \label{2.2}
In the bilinear transform, the analog filter frequency ($\Omega$) is related to the corresponding digital filter frequency ($\omega$) as
\begin{align}
\Omega = \tan \frac{\omega}{2}.
\end{align} 
 Using this relation, we obtain the analog passband and stopband frequencies as $\Omega_{p1} =0.3959 $, $\Omega_{p2} = 0.4899$ and $\Omega_{s1} = 0.3764$, $\Omega_{s2} =0.5177 $
respectively.

\section{\textbf{The IIR Filter Design}}
{\em Filter Type:}  We are supposed to design filters whose stopband is monotonic and passband equiripple.  
Hence, we use the {\em Chebyschev approximation} to design our bandpass IIR filter.

\subsection{\textbf{The Analog filter}} 
\begin{enumerate}

\item {\em Low Pass Filter Specifications:}  If $H_{a, BP}(j\Omega)$ be the desired analog band
pass filter,  with the specifications provided in Section \ref{2.2} , and $H_{a,LP}(j\Omega_L)$ 
be the equivalent low pass filter, then
\begin{equation}
\label{transition}
\Omega_L = \frac{\Omega^2 - \Omega_0^2}{B\Omega}
\end{equation}

%\begin{equation}
%H_{a, BP}(j\Omega) =  H_{a,LP}(j\Omega_L) \vert_{ \Omega_L = 
%\frac{\Omega^2 - \Omega_0^2}{B\Omega}},
%\end{equation}
where $\Omega_0 = \sqrt{\Omega_{p1}\Omega_{p2}} = 0.4404$ and $B = \Omega_{p1} - \Omega_{p2} = 0.094$. \\
 The low pass filter has the passband edge at $\Omega_{Lp} = 1$ and stopband edges at 
 \begin{align}
 \Omega_{Ls_1} = 1.5219 \\
 \Omega_{Ls_2} = -1.4775
 \end{align} 
 We choose the stopband edge of the analog low pass filter as 
 \begin{align}
 \Omega_{Ls} = \mbox{min}(\vert \Omega_{Ls_1}\vert,\vert \Omega_{Ls_2}\vert) = 1.4775.
 \end{align}

\item {\em The Low Pass Chebyschev Filter Paramters:}  The magnitude squared of the Chebyschev low pass filter is given by 
\begin{equation}
\label{lpfirst}
\vert H_{a,LP}(j\Omega_L)\vert^2 = \frac{1}{1 + \epsilon^2c_N^2(\Omega_L/\Omega_{Lp})}
\end{equation}
The passband edge of the low pass filter is chosen as $\Omega_{Lp} = 1$, (\ref{lpfirst}) may be rewritten as
\begin{equation}
\label{lpsecond}
\vert H_{a,LP}(j\Omega_L)\vert^2 = \frac{1}{1 + \epsilon^2c_N^2(\Omega_L)}
\end{equation}

Here $c_N$ denote the chebyshev polynomials for particular order N of the filter.
\begin{align}
c_N(x) &= \cosh(N \cosh^{-1}x) \\
c_0(x) &=  1 \\
c_1(x) &=  x
\end{align}
 
 There exists a recurssive relation from which all the polynomials can be found out.
\begin{align}
    c_{N+2} &= 2xc_{N+1} - c_{N}  \label{eq:cheby_poly_relation}
\end{align}
Imposing the band restrictions on \eqref{lpfirst} \\
\begin{align}
    \vert H_{a,LP}(j\Omega_L)\vert^2 < \delta_{2} \hspace{5pt} \text{for}\hspace{5pt} \Omega_L = \Omega_{Ls}\\
    1-\delta_{1}<\vert H_{a,LP}(j\Omega_L)\vert^2 < 1 \hspace{5pt} \text{for}\hspace{5pt} \Omega_L = \Omega_{Lp}
\end{align}
we obtain :
\begin{eqnarray}
\label{lpdesign}
\frac{\sqrt{D_2}}{c_N(\Omega_{Ls})} \leq \epsilon \leq \sqrt{D_1}, \nonumber \\
N \geq \left\lceil \frac{\cosh^{-1}\sqrt{D_2/D_1}}{\cosh^{-1}\Omega_{Ls}} \right\rceil,
\end{eqnarray}
where $D_1 = \frac{1}{(1 - \delta)^2}-1$ and $D_2 = \frac{1}{\delta^2} - 1$ and $\left \lceil . \right \rceil$ is known as the ceiling operator.\\ 
\input{tables/t1.tex}
The below code plots \eqref{lpfirst} for different values of $\epsilon$ .
\begin{lstlisting}
https://github.com/BATCHUISHITHA/EE-1205/blob/main/filterdesign/codes/1.py
\end{lstlisting}
\begin{figure}[H]
\centering
\includegraphics[width=1\columnwidth]{figs/1.png}
\caption{The Analog Low-Pass Frequency Response for $0.31 \leq \epsilon \leq 0.62$}
\label{fig:H_for_diff_eb}
\end{figure}

In \figref{fig:H_for_diff_eb} we can observe the equiripple behaviour in passband and monotonic behaviour in stopband. As the value of $\epsilon$ increases the value of $\vert H_{a,LP}(j\Omega_L)\vert$ decreases.\\

\item {\em The Low Pass Chebyschev Filter:} 
The next step in design is to find an expression for magnitude response in $s$ domain. 

Using $s=j\Omega$ or in this case $s_{L}=j\Omega_{L}$ we obtain:
\begin{align}
    \vert H_{a,LP}(j\Omega_L)\vert^2 = \frac{1}{1 + \epsilon^2c_N^2(\frac{s_L}{j})}
\end{align}
To find poles equate the denominator to zero:
\begin{align}
    {1 + \epsilon^2c_N^2\brak{\frac{s_L}{j}}} &=0\hspace{5pt }
    \text{where } \hspace{10pt} c_N(x) = cos\brak{Ncos^{-1}\brak{x}} \label{eq:pole_ques}
\end{align}
On solving \eqref{eq:pole_ques} we obtain poles :
\begin{align}
    s_{k} &= -\Omega_{Lp} \sin\brak{A_k}\sinh\brak{B_k} - j\Omega_{Lp}\cos\brak{A_k}\cosh\brak{B_k}
\end{align}
where $k$ is the index of the pole and \\
\begin{align}
    A_k &= \brak{2k+1}\frac{\pi}{2N}\\
    B_k &= \frac{1}{N} \sinh^{-1}\brak{\frac{1}{\epsilon}}
\end{align}

The below code computes the values of $s_k$ and stores it in a text file. 
\begin{lstlisting}
https://github.com/BATCHUISHITHA/EE-1205/blob/main/filterdesign/codes/sk_gen.c
\end{lstlisting}
The poles obtained are formulated in the table below.
\input{tables/t2.tex}

The below code plots the pole-zero plot.
\begin{lstlisting}
https://github.com/BATCHUISHITHA/EE-1205/blob/main/filterdesign/codes/2.py
\end{lstlisting}
\begin{figure}[H]
\centering
\includegraphics[width=1\columnwidth]{figs/2.png}
\caption{The Pole zero plot and all the poles lie on an ellipse. The left and right poles have been identified as shown.}
\label{fig:pole_zero_plt}
\end{figure}
The poles in the left half of the plane are considered in the design as we intend to design a stable system.\\
Therefore the magnitude response is written as :- 
\begin{align}
    H_{a,LP}(s_L) &= \frac{G_{LP}}{\brak{s_L-s_1}\brak{s_L-s_2}\brak{s_L-s_3}\brak{s_L-s_8}}
\end{align}
where $G_{LP}$ is the gain of the Low pass filter. Refer to \tabref{tab:values poles sk} for $s_k$ values.\\

We know that from \eqref{lpfirst}:-
\begin{align}
    \abs{ H_{a,LP}(s_L)} &= \frac{1}{\sqrt{1+\epsilon^2}} \text{at} \hspace{5pt} \Omega_{L}=1 \implies s_{L} = j \label{eq:Gain_eq_LP} 
\end{align}
Substituting respective values in \eqref{eq:Gain_eq_LP} we get $G_{LP}=0.4033$
\begin{align}
     H_{a,LP}(s_L) &= \frac{0.4033}{\brak{s_L-s_1}\brak{s_L-s_2}\brak{s_L-s_3}\brak{s_L-s_8}}\\
     &=\frac{0.4033}{(s_L^4 + 1.28s_L^3 + 1.82s_L^2 + 1.13s_L + 0.42)} \label{sL}
\end{align}
The below code plots DESIGN VS SPECIFICATION.
\begin{lstlisting}
https://github.com/BATCHUISHITHA/EE-1205/blob/main/filterdesign/codes/3.py
\end{lstlisting}
\begin{figure}[H]
\centering
\includegraphics[width=1\columnwidth]{figs/3.png}
\caption{Design vs Specification corresponding to \eqref{lpsecond} and \eqref{sL}}
\label{fig:designvsspecification_plt}
\end{figure}

\item {The Band Pass Chebyschev Filter:} 
After verifying design with the required specifications the next step in design is to jump to required type of filter using frequency transformation. 
\begin{align}
    s_L &= \frac{s^2 + \Omega_0^2}{Bs} \\
    H_{a,BP}(s) &= G_{BP}H_{a,LP}(s_L)\vert_{s_L = \frac{s^2 + \Omega_0^2}{Bs}},
\end{align}
As there is one to one correspondence between the filters so $\Omega=\Omega_{p1}$ should correspond to $\Omega_{Lp}$
\begin{align}
    s &= j\Omega_{p1}\\
    s_{L} &= \frac{(j\Omega_{p1})^2 + \Omega_0^2}{B(j\Omega_{p1})} \label{eq:res1} \\ 
    \abs{H_{a,BP}(j\Omega_{p1})} &= 1 \\
    G_{BP}\abs{H_{a,LP}(s_L)} &= 1 \label{eq:res2}
\end{align}
Substituting \eqref{eq:res1} in \eqref{eq:res2} we obtain Gain of required bass pass filter:
\begin{align}
    G_{BP} &= 1.047 
\end{align}
Thus the response in s domain 
\begin{align}
    H_{a,BP}\brak{s} &= \frac{3.15\times 10^{-5}s^4}{(s^8 + 0.12s^7 + 0.79s^6 + 0.07s^5 + 0.23s^4 + 0.014s^3 + 0.03s^2 + 0.0008s + 0.004)}
 \label{eq:magnitude_bandpass_sdom}
\end{align}
The expressions in the s-domain and gain factors are computed by writing a Python code. \\
In Figure 3, we plot $\vert H_{a,BP}(j\Omega)\vert$ as a function of $\Omega$ for both positve as
well as negative frequencies.  We find that the passband and stopband frequencies in the figure
match well with those obtained analytically through the bilinear transformation.\\
The below code plots the figure below
\begin{lstlisting}
https://github.com/BATCHUISHITHA/EE-1205/blob/main/filterdesign/codes/4.py
\end{lstlisting}
\begin{figure}[H]
\centering
\includegraphics[width=1\columnwidth]{figs/4.png}
\caption{The Analog Bandpass Magnitude Response from \eqref{eq:magnitude_bandpass_sdom}.The filter design specifications are satisfied}
\label{fig:band_pass_filter}
\end{figure}

subsection{The Digital Filter}
From the bilinear transformation, we obtain the digital bandpass filter from the corresponding analog filter as
\begin{align}
    H_{d,BP}(z) = GH_{a,BP}(s)\vert_{s = \frac{1-z^{-1}}{1 + z^{-1}}}
\end{align}
Substituting $s=\frac{1-z^{-1}}{1+z^{-1}}$ in \eqref{eq:magnitude_bandpass_sdom} and calculating expression using a python code we get :
\begin{align}
    H_{d,BP}(z) = \frac{z^{-8} - 4z^{-6} + 6z^{-4} - 4z^{-2} + 1}{2.2z^{-8} - 11.8z^{-7} + 31.8z^{-6} - 54.0z^{-5} + 63.1z^{-4} - 51.3z^{-3} + 28.7z^{-2} - 10.1z^{-1} + 1.8}
\end{align}

where $G=3.14\times 10^{-5}$ \\
The below code plots  the figure below
\begin{lstlisting}
https://github.com/BATCHUISHITHA/EE-1205/blob/main/filterdesign/codes/5.py
\end{lstlisting}   
\begin{figure}[H]
\centering
\includegraphics[width=1\columnwidth]{figs/5.png}
\caption{Digital Specifications are met. Passband and stopband frequencies are same}
\label{fig:Digital_BPF}
\end{figure}
\end{enumerate}

\section{The FIR Filter}
We design the FIR filter by first obtaining the (non-causal) lowpass equivalent using the Kaiser window
and then
converting it to a causal bandpass filter.
\subsection{The Equivalent Lowpass Filter}
The lowpass filter has a passband frequency $\omega_l$ and transition band $\Delta \omega = 2\pi \frac{\Delta F}{F_s} = 0.0125\pi$.
The stopband tolerance is $\delta=0.15$.The cutoff-frequency is given by :
\begin{align}
    \omega_{l} &= \frac{B}{2}\\
                &= 0.025\pi
\end{align}

The below code plots the figure below
\begin{lstlisting}
https://github.com/BATCHUISHITHA/EE-1205/blob/main/filterdesign/codes/6.py
\end{lstlisting} 
\begin{figure}[H]
\centering
\includegraphics[width=1\columnwidth]{figs/6.png}
\caption{Frequency response and impulse response of an ideal Low Pass Filter}
\label{fig:LPF_FIR_1}
\end{figure}

The impulse response of ideal Low Pass Filter is given by :
\begin{align}
    h\brak{n} = 
\begin{cases} 
    \frac{w_l}{\pi}, & \text{if } n = 0 \\
    \frac{\sin(w_l n)}{n\pi}, & \text{if } n \neq 0
\end{cases} \label{eq:h(n)_for_LPF}
\end{align}
From \eqref{eq:h(n)_for_LPF} we conclude that $h\brak{n}$ for an ideal Low Pass Filter is not causal and can neither be made causal by introducing a finite delay. And $h\brak{n}$ do not converge and hence the system is unstable.
\subsection{The Kaiser Window}
Therefore we move on windowing the impulse response.A window function is chosen and multiplied. The Kaiser window is defined as
\begin{align}
    w(n) =
    \begin{cases}
    \frac{I_0\left[ \beta N \sqrt{1 - \left(\frac{n}{N}\right)^2} \right]}{I_0(\beta N)},
\indent -N \leq n \leq N, \indent \beta > 0 \nonumber \\
 0 \hspace{2.5 cm} \mbox{otherwise,}
 \end{cases}
\end{align}

\begin{enumerate}
\item  N is chosen according to
\begin{align}
    N \geq \frac{A-8}{4.57\Delta \omega},
\end{align}
where $A = -20\log_{10}\delta$.  Substituting the appropriate values from the design specifications, we obtain
$A = 16.4782$ and $N \geq 48$.


\item  $\beta$ is chosen according to

\begin{align}
    \beta N = \left\{ \begin{array}{ll} 0.1102(A-8.7) & A > 50 \\
0.5849(A-21)^{0.4}+ 0.07886(A-21) & 21 \leq A \leq 50 \\
0 & A < 21\end{array} \right.
\end{align}
The window function is defined as :
\begin{align}
    w\brak{n} = 
\begin{cases} 
    1, & \text{for } -48\leq n \leq 48 \\
    0, & \text{otherwise } 
\end{cases} \label{eq:w(n)_for_Kaiser}
\end{align}
Therefore the desired impulse response is :
\begin{align}
    h_{lp} &= h_{n}w_{n}
\end{align}
\begin{align}
    h\brak{n} = 
\begin{cases} 
    \frac{\sin(w_l n)}{n\pi},  & \text{for } -48\leq n \leq 48 \\
    0 &\text{otherwise}
\end{cases} \label{eq:h(n)desired_for_LPF}
\end{align}
\end{enumerate}

The below code plots  the figure below
\begin{lstlisting}
https://github.com/BATCHUISHITHA/EE-1205/blob/main/filterdesign/codes/7.py
\end{lstlisting} 
\begin{figure}[H]
\centering
\includegraphics[width=1\columnwidth]{figs/7.png}
\caption{Magnitude Response of Low Pass Filter after using Kaiser Window}
\label{fig:Kaiser_LPF_response}
\end{figure}

\subsection{The Equivalent Band Pass Filter}
A Band-Pass Filter (BPF) can be obtained by subtracting the magnitude response of a Low-Pass Filter (LPF) with cutoff frequency $\omega_{p1}$ from another LPF magnitude response with cutoff frequency $\omega_{p2}$.

\begin{align}
    h_{BP}\brak{n} = 
\begin{cases} 
    \frac{\sin(w_{p2} n)}{n\pi} -\frac{\sin\brak{\omega_{p1}n}}{n\pi},  & \text{for } n\neq 0\\\
    \frac{\omega_{p2}-\omega_{p1}}{\pi} &\text{for} n= 0
\end{cases} \label{eq:h(n)desired_for_LPF1}
\end{align}
\begin{align}
     \frac{\sin(\omega_{p2} n)}{n\pi} -\frac{\sin\brak{\omega_{p1}n}}{n\pi} &= \frac{2\cos{\brak{\frac{\omega_{p2}n+\omega_{p1}n}{2}}}\sin{\brak{\frac{\omega_{p2}n-\omega_{p1}n}{2}}}}{n \pi}\\
            &= \frac{2\cos{\brak{0.265n\pi}}\sin{\brak{0.025n\pi}}}{n\pi}
\end{align}
Multipying by window function we get :
\begin{align}
    h_{BP}\brak{n} = 
\begin{cases} 
   \frac{2\cos{\brak{0.265n\pi}}\sin{\brak{0.025n\pi}}}{n\pi},  & \text{for } -48\leq n \leq 48 \\
    0 &\text{otherwise}
\end{cases} \label{eq:h(n)desired_for_LPF2}
\end{align}

The below code plots the figure below
\begin{lstlisting}
https://github.com/BATCHUISHITHA/EE-1205/blob/main/filterdesign/codes/8.py
\end{lstlisting} 
\begin{figure}[H]
\centering
\includegraphics[width=\columnwidth]{figs/8.png}
\caption{Magnitude Response of Band Pass Filter after using Kaiser Window}
\label{fig:Kaiser_BPF_response}
\end{figure}

\end{document}
